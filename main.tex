\DocumentMetadata{unactivate=tagging}
\documentclass[12pt,a4paper]{article}
\usepackage[utf8]{inputenc}
\usepackage[margin=1in]{geometry}
\usepackage{graphicx}
\usepackage{hyperref}
\usepackage{xcolor}
\usepackage{listings}
\usepackage{enumitem}
\usepackage{booktabs}

% Colors
\definecolor{azureblue}{rgb}{0,0.5,1}
\definecolor{backcolour}{rgb}{0.95,0.95,0.92}
\definecolor{codegreen}{rgb}{0,0.6,0}

% Code style
\lstset{
    backgroundcolor=\color{backcolour},
    commentstyle=\color{codegreen},
    basicstyle=\ttfamily\footnotesize,
    breaklines=true,
    numbers=left,
    numbersep=5pt,
    tabsize=2,
    frame=single
}

\title{\Huge\textbf{Climate MLOps : Guide Complet}\\
\Large Système de Prédiction Météo Autonome (Marrakech)\\
\vspace{0.5cm}
\large Architecture, Continuous Training \& Déploiement Azure}
\author{Sirgiane ouiçal}
\date{28 Décembre 2025}

\begin{document}

\maketitle
\tableofcontents
\newpage

\section{Introduction et Vision}
Le projet \textbf{Climate MLOps} n'est pas seulement un modèle de prédiction météo ; c'est un écosystème autonome capable de collecter des données, de s'auto-entraîner et de se déployer dans le Cloud. La vision centrale est le \textbf{Continuous Training} : le système s'adapte aux changements climatiques de Marrakech sans intervention humaine.

\section{Architecture Technique}
Le système repose sur une architecture multi-conteneurs (10 services) orchestrée par Docker.

\subsection{Composants Principaux}
\begin{itemize}
    \item \textbf{Base de Données} : PostgreSQL pour le stockage des relevés historiques et des métadonnées Airflow.
    \item \textbf{Orchestration} : Apache Airflow gère le pipeline de données et les cycles d'entraînement.
    \item \textbf{ML Tracking} : MLflow enregistre chaque expérience, métrique et modèle produit.
    \item \textbf{Frontend \& API} : FastAPI sert les prédictions via un Dashboard interactif.
\end{itemize}

\subsection{Image Docker Unifiée}
Pour garantir la cohérence, nous utilisons une seule image (`climate-mlops-weather-api`) basée sur \texttt{apache/airflow:2.7.3-python3.10}. Cette image contient à la fois le moteur de calcul (Scikit-learn, Optuna), l'API et l'orchestrateur.

\section{Le Cycle de Continuous Training}
Le pipeline Airflow (`climate_pipeline_dag`) est le cœur du système.

\begin{enumerate}
    \item \textbf{Collecte Quotidienne (6h00)} : Récupération des données réelles depuis l'API Open-Meteo.
    \item \textbf{Cycle de 7 jours} : Tous les 7 jours, le système lance un ré-entraînement complet.
    \item \textbf{Sélection du "Best Model"} : Trois modèles s'affrontent (Linear Regression, Random Forest, Gradient Boosting). Le gagnant est choisi via un score composite (RMSE, $R^2$, temps de calcul).
    \item \textbf{Quality Gates} : Un modèle n'est promu en production que si son $R^2 \ge 0.7$ et s'il s'améliore par rapport au modèle précédent.
\end{enumerate}

\section{Guide de Validation et Déploiement}

\subsection{Phase 1 : Validation Locale}
Avant tout déploiement Cloud, nous validons l'intégrité du système en local :
\begin{lstlisting}[language=bash]
# 1. Lancer la stack locale
make up
# 2. Verifier la sante du systeme
python3 verify_deployment.py
# 3. Executer les tests unitaires
make test
\end{lstlisting}

\subsection{Phase 2 : Préparation Azure (ACR)}
\begin{lstlisting}[language=bash]
# Connexion et definition de la souscription
az login --use-device-code
az account set --subscription "1815cb03-0ab6-4382-9f78-d03c507c84e4"

# Creation du registre
az acr create -g rg-projet -n climatemlopsreg --sku Basic --admin-enabled true

# Push de l'image validee
az acr login --name climatemlopsreg
export ACR_URL=$(az acr show -n climatemlopsreg --query loginServer -o tsv)
docker tag climate-mlops-weather-api:latest $ACR_URL/climate-mlops:latest
docker push $ACR_URL/climate-mlops:latest
\end{lstlisting}

\begin{center}
\fbox{
\begin{minipage}{0.9\textwidth}
\textbf{Bénéfices du Cloud}
\begin{itemize}
    \item \textbf{Scalabilité Automatique} : Gère les pics de charge sans intervention manuelle.
    \item \textbf{Infrastructure Gérée} : Haute disponibilité garantie par Azure (Switzerland North).
    \item \textbf{Modèle Serverless} : Paiement à l'usage réel, idéal pour l'optimisation des coûts.
\end{itemize}
\end{minipage}
}
\end{center}
\begin{lstlisting}[language=bash]
az containerapp create \
  --name weather-api \
  --resource-group rg-projet \
  --environment climate-mlops-env \
  --image climatemlopsreg.azurecr.io/climate-mlops:latest \
  --target-port 8000 \
  --ingress external \
  --cpu 1.0 --memory 2.0Gi \
  --env-vars PYTHONPATH="/workspace"
\end{lstlisting}

\section{Maintenance et Nettoyage}
Pour arrêter tous les services et libérer les ressources Azure après usage :
\begin{lstlisting}[language=bash]
# Supprimer toutes les ressources Azure du projet
az group delete --name rg-projet --yes --no-wait
\end{lstlisting}

\section{Conclusion}
Ce guide définit un workflow MLOps complet. Grâce à la conteneurisation et à l'automatisation d'Airflow, le système de Marrakech est prêt pour la production, offrant une solution robuste et scalable pour la prévision climatique.

\end{document}